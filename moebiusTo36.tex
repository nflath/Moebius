\documentclass{article}
\usepackage{graphicx}
\usepackage{url}
\usepackage{longtable}
\usepackage{mathtools}

%\addtolength{\oddsidemargin}{-.875in}
%\addtolength{\evensidemargin}{-.875in}
%\addtolength{\textwidth}{1.75in}
%\addtolength{\topmargin}{-.875in}
%\addtolength{\textheight}{1.75in}

\usepackage{tikz}
\usepackage{tikz-cd}
\usetikzlibrary{automata,positioning}

\begin{document}  
\begin{center}{\bf Calculations for Moebius: 1 through 36}\end{center}



1.  $\mu=1$. The factorization is just 1, by definition.

\smallskip
2. $\mu=-1$.  Can only be prime.

\smallskip
3. $\mu=-1$.  Can only be prime; we have not had 3 primes yet.

\smallskip
4. $\mu=0.$  First number divisible by a square. Has to be $2.2$

\smallskip
5. $\mu=-1$.   Can only be prime;  we have not had 3 primes yet.

\smallskip
6. $\mu=+1$.   Smallest with $\mu=1$ is product of two smallest primes, $2.3$.

\smallskip
7. $\mu=-1$.   Can't be $2.3.5$ since we have not even had $2.5$ yet. So 7 is prime.

\smallskip
8. $\mu=0$. Options: $2^2.2$, $3^2$.  Can't tell which.

9. $\mu=0$.  If $8=2^2.2$, options are $2^2.3$ and $3^2$ and $3^2$ is smaller because $3<2^2$.  If $8 = 3^2$, options are $2^2.2$, $3^2.2$ and $5^2$ of which $2^2.2 = 2^3$ is smallest. 

Conclusion:  8 and 9 are $2^3$ and $3^2$ but we do not know which is which.

\smallskip
10. $\mu=+1$.  Can't be a product of 4 primes since we have not even had $2.3.5$ yet, so it is a product of two primes.  Options are $2.5$  and $3.5$ (organizing by smallest prime) so $10=2.5$, the smaller.

\smallskip
11. $\mu=-1$. Prime. Not $2.3.5$ because we have not  had $3.5$.

\smallskip
12. $\mu=0$. Options are $2^2.3$, $3^2.2$ (the same as preceding), and $5^2$ (which is too big, because $2^2.3<5.5$).  So $12 = 2^2.3$.

\smallskip
13. $\mu=-1$.  Prime. Not $2.3.5$ because we have not had $3.5$ yet.

\smallskip
14. $\mu=1$.  Options are $2.7$, $3.5$, $5.7$ and it is not $5.7$ which is larger than the others, so 14 is $2.7$ or $3.5$.

\smallskip
15. $\mu=1$.  If $14=2.7$ options are $2.11$, $3.5$, $5.7$ and it can not be $2.11$ because $3.5<(2.2).5<2.11$, nor is it $5.7$ since $3.5<5.7$.  If $14 = 3.5$ options for 15 are $2.7$, $3.7$, $5.7$ the smallest of which is $2.7$.  

Conclusion: 14 and 15 are $2.7$ and $3.5$ but we do not know which is which. 

\smallskip
INTERLUDE to sort out 14 and 15.  Let $Z(x)$ be the number of integers $n$ such that $1\le n \le x$ and $\mu(n)=0$.  

Compute $Z(2.3.5)$. 

We need to list the elements $n$ such that (1) $1\le n \le  2.3.5$ and (2) $\mu(n)=0$.  
We can list them in factored form. 

First batch are of the form $2^2.a$ where $a$ an  integer such that $2^2.a\le 2.3.5$,  so $2.a\le 3.5$.  The products 
$2.1$, $2.2$, $2.3$, $2.4$, $2.5$, $2.6$ all ok.  For the last, $2.6=2.2.3<3.5$ because $2.2<5$ (we already verified this).  How about $2.7$?  We do not know, because it depends which comes first, $2.7$ or $3.5$, which is just the question we are trying to resolve.  But $2.8$ is too big, since  the possible factorizations of $2.8$, which are $2.2^3$ and $2.3^2$, are greater than $3.5$ because they are not any of the numbers 1 through 15.  So there are 6  (if $2.7>3.5$) or 7 (if $2.7<2.5$) numbers in our list so far. 

Next batch are of the form $3^2.b$ where $b$ is an integer such that $3^2.b \le 2.3.5$ and that is not on the $2^2.a$ list we already made. Thus $3.b\le2.5$. The products $3.1$, $3.2$, $3.3$ ok (we know that $3^2< 2.5$)  but $3.4=3.2^2$ is too big because $3.2^2 = 2.(2.3) >2.5$.  We have just three numbers to add to  the list of 6 or 7 we already had.

Finally, there are numbers $5^2.c$ such that $5^2.c\le 2.3.5$, or equivalently $5.c\le  2.3$, so there is only $c=1$.    

We do not need to go up to $7^2$ because $7.7 >(2.3).5$. 

So we have $Z(2.3.5) = 6+3+1 =10$ (if $2.7>3.5$)  or $7+3+1=11$ (if $2.7<3.5$)

Now we look at the known values of the moebius function.   Based on counting how many values are zero, we have 
$Z(26)= 9$, $Z(27)=10$, $Z(28) = 11$. 

We can not have $Z(2.3.5) = 10$ because that would imply $2.3.5=27$ (27 is the only number with $Z=10$), but  $\mu(2.3.7)=-1$ and  $\mu(27)=0$. Therefore  $Z(2.3.5)=11$ which we already saw implies $2.7<3.5$.

Conclusion: $14=2.7$ and $15=3.5$ and $2.3.5$ is $29$, $30$, or $31$ (the only numbers $n$ with $Z(n)=11$ and $\mu(n) = -1$. . 

\smallskip
INTERLUDE to sort out 8 and 9.

Since $3.11 > 2.3.5 \ge 29$ and $\mu(3.11)= 1$, we have $3.11 \ge 33$. Since $2^2.3^2 > 3.11\ge 33$ and $\mu(2^2.3^2) =0$,  we have $2^2.3^2 \ge 36$, and hence  $Z(2^2.3^2) \ge Z(36) = 13$. 

But we can also  figure out $Z(2^2.3^2)$ by listing its elements. 

First batch: $2^2.a$ where $1\le a\le 3^2$ so we have so far found 8 elements  if $3^2=8$ and 9 elements  if $3^2=9$. 

Next batch: $3^2.1$, $3^2$, $3^2.3$,  (but $3^2.4$ is already in the first batch, and $3^2.5$, is too big), so 3 more elements to count. 

Next batch. $5^2$.  ($5^2.2$ is too big).  This is one more element contributing to $Z(2^2.3^2)$.  

Thus: $Z(2^2.3^2) = 8+3+1= 12$  if $3^2=8$ and  $Z(2^2.3^2) =9+3+1=13$  if $3^2=9$. We already know that $Z(2^2.3^3)\ge 13$.  

Conclusion: $8=2^3$, $9=3^2$, $Z(2^2.3^2)=13$ and hence $36=2^2.3^2$.   

PROGRESS report:  We have factored everything up through 15, also factored 36, and have figured out that $2.3.5$ is one of 29, 30, or 31.  �


\smallskip
16.   $\mu = 0$.  So far for $\mu=0$ we have had $2^2=4$, $2^2.2=2^3=8$, $2^2.3=12$, and $3^2=9$.  Since 16 is the next, it must be one of $2^4.4=2^4$, $3^2.2$, or $5^2$.  We know that $2^4< 2.3^2$ (because   $2^3<3^2$)  and that $2^4=2^2.2^2<5.5=5^2$, so $16= 2^4$. 

\smallskip
17. $\mu=-1$.  Can't be $2.3.5$ because we have not had $2.3.3$ yet.  Hence 17 is prime.

\smallskip
18. $\mu=0$.  Options are $2^2.5$ or $3^2.2$ or $5^2$; but $3^2.2<  (5.2).2 = 2^2.5<5^2$ and  we have not had $2^2.5$ yet so $18 =2.3^2$. 

\smallskip
19. $\mu=-1$.  Can't be $2.3.5$ because  we have not had  $3.7$ yet, so 19 is prime. 

\smallskip
20. $\mu = 0$. Must be $2^2.5$ or $3^2.3$ or $5^2$.  We have $2^2.5 < 2^2.6 = 2^3.3 <3^2.3$ because $2^3<3^2$, and $2^2.5<5^2$, so $20=2^2.5$.
 
 \smallskip
21.   $\mu=1$.  Candidates are $2.11$, $3.7$, $5.7$. (Can't have 4 factors because we have not had $2.3.5$ yet.) Can't be $5.7$ because  $3.7< 5.7$. Hence  21 is $2.11$ or $3.7$ but we don't know which.

\smallskip
22. $\mu=1$. If  $21=3.7$ the options are $2.11$, $3.11$, $5.7$.  But $2.11<3.11$ and we have not had $5.5<5.7$ yet, so in this case $22=2.11$.  If $21=2.11$  the options are $2.13$. $3.7$, $5.7$. We have not had $5.5$ yet, so $5.7$ is ruled out. Also, $3.7<2^3.3 =2.12<2.13$ so $2.13$ is ruled out.


Conclusion:  21 and 22 are  $3.7$ and $2.11$ but we do not know which is which. 

\smallskip
23. $\mu=-1$. Prime. Can't be $2.3.5$ because we have not had $5.5<2.3.5$.

\smallskip
24. $\mu=0$. The options are $2^2.6=2^3.3$, $3^2.3=3^3$ and $5^2$. Since $2^3.3 <3^2.3$ (because $2^3<3^2$),  we see that 24  is  $2^3.3$ or $5^2$. 

\smallskip
25. $\mu=0$. If $24 = 2^3.3 =2^2.6$ the options are $2^2.7$, $3^2.3=3^3$, and $5^2$.  We have not had $2.13<2^2.7$ yet so we can remove $2^2.7$ from the list. In this case 24 is $3^3$ or $5^2$. 

If $24=5^2$ the options are $2^2.6=2^3.3$, $3^2.3=3^3$, $5^2.2$, and $7^2$.  We rule out $5^2.2>3.5.2$  and $7^2>6^2=36$.  Thus in this case 24 is  $2^3.3$ or  $3^3$. 


Conclusion: $(24, 25)$ is one of $(2^3.3, 3^3 )$, $(2^3.3, 5^2)$, $(5^2, 2^3.3 )$, $(5^2, 3^3)$.

\smallskip
26. $\mu=1$.  The options are $2.13$, $3.11$, $5.7$.  But $3.11 >3.10=3.2.5\ge 29$ and $5.7>5.6=2.3.5\ge 29$ so $26= 2.13$.


We now know that $2^3.3 =2.12<2.13=26$, so $2^3.3$ is 24 or 25.  This removes $(5^2, 3^3)$  from the list of possible pairs for $(24, 25)$. 

Conclusion:  $(24, 25)$ is one of $(2^3.3, 3^3 )$, $(2^3.3, 5^2)$, $(5^2, 2^3.3 )$,



\smallskip
27 and 28.   

Note that  $2^3.3$, $5^2$, $3^3$, and  $2^2.7$  are all less than $2.3.5$ (which is 29, 30, or 31).  Since $\mu$ is zero for these four numbers, they are all less than or equal to 28.

Hence  24, 25, 27, 28 are (in some order)  $2^3.3$, $5^2$, $3^3$, and  $2^2.7$. There are six possibilities:

Conclusion: $(24, 25, 27, 28)$ is one of the following six quadruples:

$(2^3.3, 3^3, 5^2, 2^2.7)$

$(2^3.3, 3^3, 2^2.7, 5^2)$ 

$(2^3.3, 5^2, 3^3, 2^2.7)$ 

$(2^3.3, 5^2, 2^2.7, 3^3)$ 

$(5^2, 2^3.3, 3^3, 2^2.7)$

$(5^2, 2^3.3, 2^2.7, 3^3)$.


\smallskip
29, 30, 31. $\mu= -1$. One of them is $2.3.5$. The other two are prime because the next product of 3 primes is $2.3.7 > 2.3.6 =2^2.3^2= 36$.

\smallskip
32.  $\mu=0$.  The options  are $2^2.8=2^5$, $3^2.4=36$, $5^2.2>4^2.2= 2^5$, $7^2>6^2=36$. All but $2^5$ are too large, so $32=2^5$.

\smallskip
33, 34, 35.  $\mu=1$. Can't be a product of 4 primes $2.3.5.7$ because we have not even had $2.3.7$.  These three numbers must include the factorizations $3.11<3.12=36$, $2.17<2.18=36$ and another product of 2 primes. The next products on the list  are $2.19>2.18=36$, $3.13>3.12=36$, and $5.7$, so 
  $5.7$ must be one of 33, 34, 35. 

Conclusion:  33, 34,and 35 are $3.11$, $2.17$, and $5.7$ in some but some (as yet unknown)  order.

\smallskip
36. $\mu=0$.  We already  know that $36=2^2.3^2$. 


HOW MANY BRANCHES SO FAR? 

How many branches through 36? 

\smallskip
21 and 22:     2 options:    3.7 and 2.11 in either order.

\smallskip
24, 25, 27, 28:    6 options listed above.

\smallskip
29, 30, 31:   3 options:  prime, prime, 2.3.5  where 2.3.5 can be in any of the three positions. 

\smallskip
33, 34, 35:   6 options: $3.11$, $2.17$, $5.7$ in any order. 

TOTAL $2*6*3*6 = 216$ options. 



\end{document}